\chapter{Interactions networks}
\label{chap:prelim}

	\section{Sequence and network data}

		\subsection{Genome, transcriptome, proteome}

		\subsection{Modeling the living, biological networks}
			Biological networks are abstract representations of biological entities interconnected by some criteria.
			They can represent for example the relationships between species inside an ecosystem, or interconnections between cell types in any multicellular organism.

			In this work, we are mostly interested in networks at the biomolecular level.
			Many such networks exists, to name a few:
			\begin{itemize}
				\item \emph{Metabolic networks} represent biochemical reactions between substrates, enzymes and metabolites, and cluster them into pathways,
				\item \emph{Gene co-expression networks} represent the similarity of expression between genes in some biological setup, by interconnecting pairs of genes simultaneously expressed,
				\item \emph{Gene regulatory network} represent the indirect regulatory actions of genes, from proteins and transcription factors to gene expression levels,
				\item \emph{Protein-protein interaction networks} represent interactions between two proteins, usually of the same species.
			\end{itemize}

			On the one hand biological networks can be seen as observed or inferred facts, where the network represents the knowledge ; e.g. a known pathway that connect chemical reactants and products through enzymes.
			On the other hand they can be seen as an abstract representation of knowledge where nodes represent entities and edges represent some form of deduced connections ; XXX e.g. a gene co-expression network which can be constructed from the control and condition expression profiles of the genes, with a statistical inference over the two samples resulting in the presence or absence of the edges. (trop embrouille -- Macha) XXX

			Let us stress the importance of biological networks in modern biology.
			They structure our understanding of biological systems in such ways that both allow a comprehension of biological processes at the system level, and permit automated processing of the knowledge that they represent.
			As automated processing enabling tools, they can serve as both knowledge bases for local decisions and as global networks that can serve as XXX substrate (de quoi parles-tu? -- Macha) XXX for integrated analysis.

			XXX.

			The most fitting abstraction for those biological networks are discrete mathematics' graphs (XXX to be formally defined in ...XXX).

			Protein-protein interactions (PPI) networks play an important role in this work, and we will present them in more detail.
	
			\paragraph{Protein-Protein Interactions}

		\subsection{Gene expression}

			\paragraph{Measuring gene expression levels}

			\paragraph{Differential analysis}

				\begin{itemize}
					\item better understanding of cellular processes
					\item biomarkers discovery
				\end{itemize}

	%	\subsubsection{??? Phage display ???}
	%	\subsubsection{??? Mass spectrometry ???}

	\section{Elements of graph theory}

		\subsection{Graphs}
			Let us recall some basic material related to graphs.
			A graph $G = (V,E)$ consists of a set of vertices $V$ and a set of edges (unordered pairs of vertices) $E$.
			%To shorten the exposition, we shall usually abbreviate $|V|$ and $|E|$ to $n$ and $m$, respectively.
			We say that $G$ is node-weighted if a function $w\colon V \to \mR$ is provided.
			%A graph is a \textit{tree} if it is both connected -- there exists a path between any pair of vertices -- and acyclic -- it does contain a closed path in which the first and the last vertices are the same. 
			Given a graph $G = (V, E)$, its subgraph $G' = (V', E')$ is said to be \emph{induced} if $G'$ has exactly the edges that appear in $G$ over the vertex set $V' \in V$, that is $E' = \Set{(x, y) \in E}{x,y \in V'}$.
			We  denote the graph \emph{induced} by the node set $V'$ in $G$ by $G\left[V'\right]$.

%			\paragraph{Minimum cut}

	\section{Combinatorial optimization}
		\paragraph{Dynamic programming}
		\paragraph{Decision trees, Branch and bound, Branch and cut}
		\paragraph{Linear programming, Mixed integer linear programming}

		\subsection{Complexity}
			\paragraph{APX-hardness}
				\label{par:m3sat}
				One well studied APX-hard problem is the \msat{} problem and is defined by \textcite{papadimitriou1991optimization} as follows.
				Given a collection $C_q = \{c_1, \ldots c_q\}$ of $q$ clauses where each clause consists of a set of three literals over a finite set of $n$ boolean variables $V_n = \{x_1, \ldots x_n\}$ and every literal occurs in at most $B$ clauses, is there a truth assignment of $V_n$ satisfying the largest number of clauses of $C_q$?
	
			\paragraph{Pseudo-polynomial time}

%		\subsection{String}
%			\paragraph{Suffix trees and array}
