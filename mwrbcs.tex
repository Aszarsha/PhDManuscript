\chapter{The Ratio-Bounded MWCS}
\label{chap:rbmwcs}

	In this chapter, XXX explain XXX.

	A slightly more general version\footnote{the contribution function is more expressive here} of the problem introduced at the end of the previous chapter is called the \textsc{ratio-bounded maximum-weight connected subgraph} (\rbmwcs) problem, and in the general case is defined formally as follows.

	\textbf{\rbmwcs{}}: Given a node-weighted graph $G = (V, E)$, its node-weighting function $w\colon V \to \mR$, its contribution function $c\colon V \to \mR$, a ratio $\alpha \in [0,1]$ and a constant $C \in \mR$, find a subset $V^* \subseteq V$ such that:
	\begin{enumerate}
		\item the induced graph $G\left[V^*\right]$ is connected, and
		\item the ratio of the sum of contributions plus some constant over the number of nodes in the solution is greater than or equal to $\alpha$, that is:\\
			$\sum\limits_{v \in V^*}{c(v)} + C \geq \alpha\times\card{V^*}$, and
		\item $\sum\limits_{v \in V^*}{w(v)}$ is maximum.
	\end{enumerate}

	\begin{proposition}
		\rbmwcs{} is as difficult as \mwcs{}.
	\end{proposition}
	\begin{proof}
		Indeed, when $\forall v\in V, c(v) = 1$, the ratio $\sum\limits_{v \in V^*}{c(v)} / \card{V^*} = 1 \geq \alpha$, and the \mwcs{} and \rbmwcs{} problems are equivalent.
	\end{proof}

	\section{A more general variant of the \textsc{budget-constrained mwcs}}
		The \textsc{budget-constrained maximum-weight connected subgraph}, also named the \textsc{budgeted mwcs} (\bcmwcs{}) is formally defined as follow.
		
		\textbf{\bcmwcs{}}: Given a node-weighted graph $G = (V, E)$, its node-weighting function $w\colon V \to \mR$, its cost function $\text{cost}\colon V \to \mR^+$, and a budget $B \in \mR^+$, find the subset $V^* \subseteq V$ such that:
		\begin{enumerate}
			\item the induced graph $G\left[V^*\right]$ is connected, and
			\item the sum of the costs does not exceed the allocated budget $B$, that is:\\
				$\sum\limits_{v \in V^*}{\text{cost}(v)} \leq B$, and
			\item $\sum\limits_{v \in V^*}{w(v)}$ is maximum.
		\end{enumerate}

		Note that appart from the cost and ratio constraints, the two problems are stricly equivalent.

		\begin{proposition}
			The ratio constraint $\sum\limits_{v \in V^*}{c(v)} + C \geq \alpha\times\card{V^*}$ of the \rbmwcs{} problem is a generalization of the costs constraint $\sum\limits_{v \in V^*}{\text{cost}(v)} \leq B$ of the \bcmwcs{} problem.
		\end{proposition}

		\begin{proof}
			\begin{align*}
				\sum\limits_{v \in V^*}{c(v)} + C &\geq \alpha\times\card{V^*} \\
				\sum\limits_{v \in V^*}{c(v)} + C &\geq \sum\limits_{v \in V^*}{\alpha} \\
				\sum\limits_{v \in V^*}{c(v)} - \sum\limits_{v \in V^*}{\alpha} &\geq -C \\
				\sum\limits_{v \in V^*}{c(v) - \alpha} &\geq -C \\
				\sum\limits_{v \in V^*}{\alpha - c(v)} &\leq C \\
				\sum\limits_{v \in V^*}{c'(v)} &\leq C \\
			\end{align*}

			The range of the function $c'$ is a superset of the range of the function $\text{cost}$ since the later is positive.
			The same argument is made for \bcmwcs{}'s constant $B$ which is positive where the constant $C$ is real.
		\end{proof}

		\begin{proposition}
			There exists a linear reduction from \bcmwcs{} to \rbmwcs{} problem, hence the latter is more general.\qed{}
		\end{proposition}

		\begin{proof}
			Given an \bcmwcs{} problem,
			\begin{enumerate}
				\item the graph $G = (V, E)$ do not change, neither does the weight function $w$,
				\item define the cost function $c(v) = \alpha - \text{cost(v)},\,\,\forall \alpha \in [0,1]$, and
				\item the constant $C = B$.
			\end{enumerate}
			This defines a suitable reduction from \bcmwcs{} to \rbmwcs{}.
		\end{proof}


	\section{Pseudo-polynomial algorithms from \mwcs{}}

		XXX Argument about the reasoning with $c\colon V \to \mZ$ instead of $c\colon V \to \mR$, which 1] is valid for our reduction in chap 4 and 2] holds for \bcmwcs{} XXX

		\subsection{Explicit algorithms for trees and cactii}

		\subsection{For bounded treewidth graphs}
