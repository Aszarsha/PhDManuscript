\chapter*{Abstract}
\addcontentsline{toc}{chapter}{Abstract}
\chaptermark{Abstract}
\label{chap:abstract}

	\begin{itemize}
		\item genomic, transcriptomic
		\item sequencing
		\item database construction (networks ?)

		\item bioinformatics pipeline
		\item type of data
		\item integrative approaches
	\end{itemize}

\paragraph*{This thesis is structured in three main parts.}


In \cref{pt:1} we present an outline of the domain, first with a general overview, and later with an analysis of the state of the art.
\Cref{chap:prelim} is separated in two main sections, introducing some biology notions and some computer science background relating to this work.
\Cref{chap:state} presents a thorough review of the existing work and main methods that either serve as direct support or are relevant for our work for both module discovery and complexity analysis of similar problems.

\Cref{pt:2} is dedicated to our methodological and software contribution. It is mainly constituted of \cref{chap:xheinz}, which is sectionned as follows.
\Cref{sec:weights} establishes the basis for module discovery, assigning weights to genes based on experimental evidence.
This chapter describes two methods for weight assignment, from transcriptomic expression profiles, and from secondary evidence obtained through proteomic analysis.
\Cref{sec:mip} introduces an important algorithmic contribution to the module detection problem by describing a robust mathematical model, serving as our definition of cross-species module, with flexible requirement of sequence conservation.
Multiple techniques could be used to express and resolve this model, which implicitly includes the Maximum-Weight Connected Subgraph problem.
It will first be expressed as a Mixed-Integer Linear program, and we will present an in-depth algorithmic implementation to solve the problem in the general case.
\Cref{sec:xhres} presents our biological results and empirical analysis of algorithmic performance.

Finally, \cref{pt:3} is dedicated to the algorithmic complexity analysis of the combinatorial optimization problem underlying our model.
\Cref{chap:hard} provides detailed proofs for the frontier of difficulty for the problem introduced by our model, producing a precise separation of complexity classes between types of input graphs.
\Cref{chap:rbmwcs} describes a general scheme to construct efficient, pseudo-polynomial algorithms for a subclass of our initial problem.
