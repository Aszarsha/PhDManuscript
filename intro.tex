\chapter{Introduction}

\section{Context}

	\begin{itemize}
		\item bioinformatics pipeline
		\item type of data
		\item more integrative approaches
	\end{itemize}

\section{Preliminary notions}

	\subsection{Small biology overview}

		\subsubsection{From genome to proteins, the Central Dogma}
			\paragraph{Genome}
			\paragraph{Transcriptome}
			\paragraph{Proteome}

		\subsubsection{Modeling the living, biological networks}
			Biological networks are abstract representations of biological entities interconnected over some criteria.
			It can represent for example the relationships between species inside an ecosystem, or interconnections between cell types in any multicellular organism.

			In this work, we are mostly interested in biological networks that pertain to the living, i.e. networks of components of an organism. Many such networks exists, to name a few:
			\begin{itemize}
				\item \emph{Metabolic networks} represent biochemical reactions between substrates, enzymes and metabolites, and cluster them into pathways,
				\item \emph{Gene co-expression networks} represent the similarity of expression between genes in some biological setup, by interconnecting pairs of genes similarly expressed,
				\item \emph{Protein-protein interaction networks} represent the interaction between two proteins, usually of the same species.
			\end{itemize}

			On the one hand biological networks can be seen as observed or inferred facts, where the network is the knowledge in itself ; e.g. a known pathway that connect chemical reactants and products through enzymes.
			On the other hand they can be seen as an abstract representation of knowledge where nodes represent entities and edges represent some form of deduced connection ; e.g. a gene co-expression network which can be constructed from the control and condition expression profiles of the genes, with a statistical inference over the two samples resulting in the presence or absence of the edges.

			Protein-protein interaction (PPI) networks play an important role in this work, and we will present them in more detail.
			But first let us stress the importance of biological networks in modern biology.
	
			They structure our understanding of biological systems in such ways that both allow a comprehension of biological processes, and permit automated processing of the knowledge that they represent.
			As automated processing enabling tools, they can serve as both knowledge bases for local decisions and as global networks that can serve as substrate for integrated analysis.

			XXX.

			\paragraph{Protein-Protein Interactions}

		\subsubsection{Measuring the state of the living}

			\paragraph{Measuring gene expression levels}

			\paragraph{Differential analysis}

				\begin{itemize}
					\item better understanding of cellular processes
					\item biomarkers discovery
				\end{itemize}

	%	\subsubsection{??? Phage display ???}
	%	\subsubsection{??? Mass spectrometry ???}

	\subsection{Some computer science elements}

	\subsubsection{Graphs}
%		\paragraph{Minimum cut}

	\subsubsection{Combinatorial optimization}
		\paragraph{Dynamic programming}
		\paragraph{Decision trees, Branch and bound, Branch and cut}
		\paragraph{Linear programming, Mixed integer linear programming}

	\subsubsection{Complexity}
		\paragraph{APX-hardness}
		\paragraph{Pseudo-polynomial time}

	\subsubsection{String}
		\paragraph{Suffix trees and array}

